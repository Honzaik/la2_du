\documentclass[12pt, a4paper]{article}
\usepackage[utf8x]{inputenc}
\usepackage{indentfirst} %indentace prvního odstavce
\usepackage{mathtools}
\usepackage{amsfonts}
\usepackage{amsmath}
\usepackage{amssymb}
\usepackage{graphicx}
\usepackage{enumitem}
\usepackage{subfig}
\usepackage{float}
\usepackage[czech]{babel}

\begin{document}

\section{}
Rozšířme vektor $x(t)$ o jednu složku, která bude zaznamenávat vzdálenost od původní pozice v čase $t$. Vektor $x(t)$ má tedy tvar \\$x(t)=(odchylka, rychlost, vzdalenost)^T$, kde vzdálenost je v metrech. Kvůli tomu musíme rozšířit matice, aby to dávalo smysl. Získáme tedy rovnici:
\begin{gather*}
A = \begin{pmatrix}
1.1 & -0.5 & 0\\
0 & 1 & 0\\
0 & 0.01 & 1
\end{pmatrix},
B = \begin{pmatrix}
-2\\
0.01\\
0
\end{pmatrix}\\
x(t+1)= Ax(t) + Bu(t)
\end{gather*}

Matice poslední řádek matice $A$ počítá změnu polohy. Ze zadání víme, že frekvence kroků je 100 za 1s, takže vzdálenost se mění vztahem $1x(t)_3 + \frac{1}{100}x(t)_2$. Rychlost $x(t)_2$ nám říká kolik metrů ujede robot za jednu sekundu, takže za $\frac{1}{100}$s (jeden krok) ujede přesně jednu setinu rychlosti metrů, což přičteme k vzdálenosti $x(t)_3$. Stav $x(0)$ samozřejmě rozšíříme o jednu složku s hodnotou 0 (na začátku je vzdálenost od počátku 0).

Celkem máme tedy 100 kroků (povolena 1s ovládání a frekvence kroků je 100 za sekundu). Chceme tedy zajistit, aby $x(100)=(0,0)^T$ a zároveň, aby změny zrychlení ($0 \leq t \leq 99: u(t)-u(t-1)$) byly minimální.

Dosazením do rekurentního vzorce dostaneme vzorec pro výpočet $x(100)$
\begin{gather*}
x(100)=A^{100} x(0) + A^{99}Bu(0) + A^{98}Bu(1) + \dots + ABu(98) + Bu(99) 
\end{gather*}

Chceme $x(100)=(0,0)^T$ a hodnota $A^{100}x(0)$ je konstantní (nezávisí na $u(t)$), takže máme
\begin{gather*}
-A^{100} x(0) = + A^{99}Bu(0) + A^{98}Bu(1) + \dots + ABu(98) + Bu(99) 
\end{gather*}

Označme $C \coloneqq (A^{99}B|A^{98}B|\dots|AB|B),\ b \coloneqq -A^{100}x(0),\\ u \coloneqq (u(0), \dots, u(99))^T$. Takže máme
\begin{gather*}
Cu=b
\end{gather*}

Potřebujeme ale přetransformovat vektor $u$, aby nám řešení s nejmenší euklidovskou normou (pomocí pseudoinverzu) minimalizovalo naší upravenou normu ($0 \leq t \leq 99: u(t)-u(t-1)$). Označme $d(0)\coloneqq u(0),\ \forall t \in \{1,\dots,99\}: d(t) \coloneqq u(t)-u(t-1)$. Zřejmě platí $u(t)=\sum_{i=0}^t d(t)$. Označíme-li tedy $d \coloneqq (d(0),\dots,d(99))^T$, tak platí
\begin{gather*}
T = \begin{pmatrix}
1 & 0 & 0 & 0 & \dots & 0\\
1 & 1 & 0 & 0 & \dots & 0\\
1 & 1 & 1 & 0 & \dots & 0\\
\vdots & \vdots & \vdots & \ddots &\ddots & \vdots\\
1 & 1 & 1 & 1 & \dots & 1
\end{pmatrix}, T \in \mathbb{R}^{100 \times 100} \text{ a platí } u = Td
\end{gather*}

Nyní pokud dosadíme do původní rovnice, tak máme $(CT)d=b$. Spočítáme řešení s nejmenší normou $d_{opt}$ (minimalizuje rozdíly $u(t)-u(t-1)$) pomocí pseudoinverze neboli $d_{opt} = (CT)^{\dagger}b $, kde $(CT)^{\dagger}$ značí pseudoinverz matice $CT$. Hodnoty $u_{opt}=(u(0)_{opt},\dots,u(99)_{opt})^T$ získáme z vzorce výše, tedy $u_{opt}=Td_{opt} = T(CT)^{\dagger}b$.

\section{}
neřešil jsem

\end{document}