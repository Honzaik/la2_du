\documentclass[12pt, a4paper]{article}
\usepackage[utf8]{inputenc}
\usepackage{indentfirst} %indentace prvního odstavce
\usepackage{mathtools}
\usepackage{amsfonts}
\usepackage{amsmath}
\usepackage{amssymb}
\usepackage{graphicx}

\begin{document}

\section{}
.

\section{}
.

\section{}
\subsection{}
Z definice násobení matic a vlastností sum:
\begin{gather*}
tr(AA^T) = \sum_{i=1}^{m}(\sum_{j=1}^{n} a_{i,j} \cdot a_{i,j}) = \sum_{j=1}^{n}(\sum_{i=1}^{m} a_{i,j} \cdot a_{i,j}) = tr(A^TA) \\
\implies \\
\sum_{j=1}^{n}(\sum_{i=1}^{m} a_{i,j} \cdot a_{i,j}) = \sum_{j=1}^{n}\sum_{i=1}^{m} a_{i,j}^2 = {\Vert A \Vert}_F^2
\end{gather*}
kde $i$ je index řádku ve výsledné matici a v závorce je prvek na místě $i,i$ ve výsledné matici.
\subsection{}
Obdobně jako 3.1
\[ B \in \mathbb{R}^{n \times m}, tr(AB)= \sum_{i=1}^{m}(\sum_{j=1}^{n} a_{i,j} \cdot b_{i,j}) = \sum_{i=1}^{m}(\sum_{j=1}^{n} b_{i,j} \cdot a_{i,j}) = \sum_{j=1}^{n}(\sum_{i=1}^{m} b_{i,j} \cdot a_{i,j}) = tr(BA)\]

\subsection{}
Dokážeme rovnosti 2. mocnin, což je ekvivalentní, jelikož norma je vždy kladná.
\begin{gather*}
\Vert UAV^T \Vert^2 \stackrel{\text{3.1}}{=} tr(UAV^{T}VA^{T}U^{T}) \stackrel{\text{V má ortonormální posl. sloupců}}{=} \\
tr(UAA^{T}U^{T}) \stackrel{\text{3.1}}{=} tr(A^{T}U^{T}UA) \stackrel{\text{U má ortonormální posl. sloupců}}{=} tr(A^TA) = \\
\Vert A \Vert_{F}^2
\end{gather*}

\section{}
\subsection{}
reflexivita:
\begin{gather*}
U \preceq U \iff \forall x \in \mathbb{R}^n: x^T(U - U)x \geq 0 \implies x^T\mathit{0}x=0  \; \forall x \in \mathbb{R}^n
\end{gather*}

tranzitivita:
\begin{gather*}
U,V,W \in \mathbb{R}^{n \times n}: U \preceq V \land V \preceq W \iff \forall x \in \mathbb{R}^n: x^T(V-U)x \geq 0 \land x^T(W-V)^x \geq 0\\
x^T(W-U)x = x^T(W-V+V-U)x \stackrel{\text{vlastnosti mat. nás.}}{=}\\
x^T(W-V)x + x^T(V-U)x \stackrel{\text{1. řádek}}{\geq} 0\ \forall x \in \mathbb{R}^n \implies U \preceq W
\end{gather*}

slabá antisymetrie:
\begin{gather*}
U,V \in \mathbb{R}^{n \times n}: U \preceq V \land V \preceq U \iff V-U \ \text{poz. semidef.} \land U-V \ \text{poz. semidef.} 
\end{gather*}
Víme, že pro pozitivně semidefinitní matice platí, že jsou ortogonálně diagonalizovatelné a jejich vlastní čísla jsou nezáporná. Označme $A \coloneqq U - V$
Nechť $\lambda \in \mathbb{R}$ je vlastní číslo $A$ a $v \in \mathbb{R}^n$ jemu příslušný vlastní vektor. Platí:
\begin{gather*}
Av=\lambda v \iff -Av = - \lambda v.
\end{gather*}
Víme že $\lambda \geq 0$ díky tomu, že je to vlastní číslo $A$, což je poz. semidef. matice. Ale $-A = V-U$, což je také semidefitní matice, má evidentně $-\lambda$ jako své vlastní číslo. Takže $0 \geq \lambda \geq 0 \implies \lambda = 0$. Tedy libovolné vlastní číslo matice $A$ je 0. Z diagonalizace víme, že $A = BDB^T$, kde $D$ je diagonální (B ortogonální) a na diagonále jsou všechna vlastní čísla $A$. Tudíž $D=0 \implies A=0 \iff U = V$.

\subsection{}
\begin{gather*}
C = diag(c_1,\dots, c_n),\ D = diag(d_1,\dots, d_n),\ D-C=diag(d_1-c_1,\dots, d_n-c_n)\\
C \preceq D \iff \forall (x_1,\dots, x_n)^T \in \mathbb{R}^n: x^T(D-C)x \geq 0 \iff \\
\forall (x_1,\dots, x_n)^T \in \mathbb{R}^n: x_1^2 \cdot (d_1-c_1) + \dots + x_n^2 \cdot (d_n-c_n) \geq 0
\end{gather*}
$x$ je libovolný vektor, tedy můžeme vzít $x = e_i \ \forall i \in \{1,\dots,n\}$ a vidíme, že musí platit $\forall i \in \{1,\dots,n\}: d_i-c_i \geq 0 \iff d_i \geq c_i$. Naopak pokud je tato podmínka splněna tak rovnost bude vždy nezáporná, jelikož jsou tam druhé mocniny násobnené nezáporným číslem. Tedy $C \preceq D \iff \forall i \in \{1,\dots,n\}: d_i \geq c_i$

\subsection{}
Matice $U,V$ jsou symetrické, tedy nám udávají symetrické bilineární formy. Tvrzení 11.34 nám říká, protože je $U$ pozitivně definitní, že existuje matice $Q$ ortogonální tž.:
\begin{gather*}
c_i, d_i \in \mathbb{R}^+_0, \ C=diag(c_1,\dots, c_n), \ D = diag(d_1,\dots, d_n) : U=QCQ^T \land V = QDQ^T\\
U \preceq V \iff \forall x \in \mathbb{R}^n: x^T(V-U)x \geq 0 \iff x^TVx \geq x^TUx \iff\\
x^T(QDQ^T)x \geq x^T(QCQ^T)x\\
Q \text{ je bijekce tedy označíme-li } y \coloneqq Q^Tx \implies\\
\forall y \in \mathbb{R}^n: y^TDy \geq y^TCy \text{ tedy speciálně pro } y=e_i \ \forall i \in \{1, \dots, n\} \implies\\
\forall i \in \{1, \dots, n\}: d_i \geq c_i 
\end{gather*}

Z 1.3 víme, že $\Vert U \Vert_F = \Vert QCQ^T \Vert_F = \Vert C \Vert_F$ a $\Vert V \Vert_F = \Vert QDQ^T \Vert_F = \Vert D \Vert_F$. Díky čemu plyne $\Vert D \Vert_F \geq \Vert C \Vert_F \implies \Vert V \Vert_F \geq \Vert U \Vert_F$.
\end{document}