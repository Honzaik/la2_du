\documentclass[12pt, a4paper]{article}
\usepackage[utf8]{inputenc}
\usepackage{indentfirst} %indentace prvního odstavce
\usepackage{mathtools}
\usepackage{amsfonts}
\usepackage{amsmath}
\usepackage{amssymb}
\usepackage{graphicx}

\begin{document}

\section{}
.

\section{}
.

\section{}
\subsection{}
Z definice násobení matic a vlastností sum:
\begin{gather*}
tr(AA^T) = \sum_{i=1}^{m}(\sum_{j=1}^{n} a_{i,j} \cdot a_{i,j}) = \sum_{j=1}^{n}(\sum_{i=1}^{m} a_{i,j} \cdot a_{i,j}) = tr(A^TA) \\
\implies \\
\sum_{j=1}^{n}(\sum_{i=1}^{m} a_{i,j} \cdot a_{i,j}) = \sum_{j=1}^{n}\sum_{i=1}^{m} a_{i,j}^2 = {\Vert A \Vert}_F^2
\end{gather*}
kde $i$ je index řádku ve výsledné matici a v závorce je prvek na místě $i,i$ ve výsledné matici.
\subsection{}
Obdobně jako 3.1
\[ B \in \mathbb{R}^{n \times m}, tr(AB)= \sum_{i=1}^{m}(\sum_{j=1}^{n} a_{i,j} \cdot b_{i,j}) = \sum_{i=1}^{m}(\sum_{j=1}^{n} b_{i,j} \cdot a_{i,j}) = \sum_{j=1}^{n}(\sum_{i=1}^{m} b_{i,j} \cdot a_{i,j}) = tr(BA)\]

\subsection{}
Dokážeme rovnosti 2. mocnin, což je ekvivalentní, jelikož norma je vždy kladná.
\begin{gather*}
\Vert UAV^T \Vert^2 \stackrel{\text{3.1}}{=} tr(UAV^{T}VA^{T}U^{T}) \stackrel{\text{V má ortonormální posl. sloupců}}{=} \\
tr(UAA^{T}U^{T}) \stackrel{\text{3.1}}{=} tr(A^{T}U^{T}UA) \stackrel{\text{U má ortonormální posl. sloupců}}{=} tr(A^TA) = \\
\Vert A \Vert_{F}^2
\end{gather*}
\end{document}