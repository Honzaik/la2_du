\documentclass[12pt, a4paper]{article}
\usepackage[utf8]{inputenc}
\usepackage{indentfirst} %indentace prvního odstavce
\usepackage{mathtools}
\usepackage{amsfonts}
\usepackage{amsmath}
\usepackage{amssymb}
\usepackage{graphicx}
\usepackage[czech]{babel}

\begin{document}

\section{}
Definujeme vektor stavu HP pokémonů jako $h(t) = (h_1(t), h_2(t))^T$, kde $h_1(t)$ značí hodnotu HP Pikachu po kole $t$ a $h_2(t)$ hodnotu HP Charmandera po kole $t$. Víme, že $h_1(0)=100$ a $h_2(0)=50$. Ze zadání je zřejmé, že platí:
\begin{gather*}
A = \begin{pmatrix}
0.95 & -0.2\\
-0.1 & 1.1
\end{pmatrix}\\
h(t+1) = A \cdot h(t)
\end{gather*}

Vidíme, že toto je předpis pro lineární dynamický systém a tedy zřejmě platí $h(t) = A^{t}\cdot h(0)$.
Matice A je diagonalizovatelná, protože její vlastní vektory tvoří bázi $\mathbb{R}^2$. Platí:
\begin{gather*}
R = \begin{pmatrix}
0.647994 & -0.920202\\
-0.761646 & -0.391445
\end{pmatrix},
D = \begin{pmatrix}
1.18508 & 0\\
0 & 0.864922
\end{pmatrix}\\
A = R \cdot D \cdot R^{-1}
\end{gather*}

Matice $R$ je matice přechodu z báze $B$ tvořené vlastnímy vektory $A$ ke kanonické bázi ($K$) a $R^{-1}$ naopak. Z lineární algebry víme, že platí $[h(t)]_B = D^t \cdot [h(0)]_B$, kde $[\cdot]_B$ značí souřadnice vektorů v bázi B. Dále platí $h(t) = [h(t)]_K= R \cdot [h(t)]_B$ a $[h(0)]_B = R^{-1} \cdot [h(0)]_K = h(0)$. Dosadíme-li:
\begin{gather*}
h(t) = R \cdot [h(t)]_B = R \cdot D^t \cdot [h(0)]_B =  R \cdot D^t \cdot R^{-1} \cdot h(0) \text{ kde }\\
\end{gather*}
Po dosazení a vyčíslený máme explicitní vzorec pro $h(t)$:
\begin{gather*}
h(t) = \begin{pmatrix}
h_1(t)\\
h_2(t)
\end{pmatrix}\approx
\begin{pmatrix}
104.661 \cdot 0.864922^t - 4.66082 \cdot 1.18508^t\\
44.5217 \cdot 0.864922^t - 5.47828 \cdot 1.18508^t\\
\end{pmatrix}
\end{gather*}

Po 9. kole ($t=9$) má Pikachu HP $h_1(9) \approx 6.864$ a Charmander $h_2(9) \approx 37.317$. V po 10. kole by bylo Pikachu mrtvé s $h_1(10) \approx -0.942$. Potřebujeme utéct po 9. kole.
\end{document}