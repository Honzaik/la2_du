\documentclass[12pt, a4paper]{article}
\usepackage[utf8]{inputenc}
\usepackage{indentfirst} %indentace prvního odstavce
\usepackage{mathtools}
\usepackage{amsfonts}
\usepackage{amsmath}
\usepackage{amssymb}
\usepackage{graphicx}
\usepackage[czech]{babel}
\begin{document}


Tématem dnešní politologické rubriky jsou zajímavé korelace. Korelace je matematický termín, který označuje míru závislosti nějakých 2 věcí sobě. Jelikož jsme si všichni prošli Školou života, tak víme, že korelace implikuje kauzalitu. Korelaci, na kterou se dnes podíváme, bude popularita termínů \uv{toaletní papír} a \uv{politika}. Potřebujeme si nějak vysvětlit, proč to tak je. Existuje model, který to předpovídá, takže to musí být pravda.

V dnešní době se na internetu nenakupují pouze věci jako elektronika apod., ale i věci co si kupujeme každodenně. Jednou z těch věcí je samozřejmě toaletní papír. Příčina této závislosti je snad nyní zřejmá stejně jako většina věcí, co předpokládají za zřejmé vyučující na jedné nejmenované vysoké škole.

Po nakoupení toaletního papíru se český občan již nemusí bát vyhledávat detaily o české politické situaci. Zaskočí-li ho něco, tak se nemusí obávat nedostatku toaletního papíru doma na záchodě. Tohoto fenoménu si nedávno všimnuli provozovatelé podniků, ve kterých se často řeší politická situace. Takovými podniky jsou hlavně hospody a kavárny. Provozovatelé si vytipují do budoucna dny spojené s politikou, a poté nakoupí dostatek toaletního papíru, aby ho byl dostatek, jelikož znají své štamgasty.

Nejznámějším dohledatelným případem této „nedostatek toaletního papíru kvůli politické situaci“ krize byla tzv. Velká toaletní krize v Pražské kavárně na konci ledna 2018. Na konci ledna 2018 se konalo druhé kolo volby prezidenta České republiky. Český lid vybíral mezi Milošem Zemanem a Jiřím Drahošem. Je známo, že Pražská kavárna a její pravidelní zákazníci, nemají v oblibě Miloše Zemana. Miloš Zeman také několikrát dal najevo, prostřednictvím svého mluvčího, který mu je věrnější, než byl pes Hachikō svému páníčkovi, že lidi, co navštěvují Pražskou kavárnu (tzv. „lepšolidi“), také nemá rád. Když byl tedy Miloš Zeman zvolen prezidentem, tak za pár minut nebyla v Pražské kavárně jediná role toaletního papíru (popravdě zbyla jedna role jednovrstvého toaletního papíru, ale ten samozřejmě nikdo nechce používat).

\end{document}